\documentclass{article}
\usepackage[utf8]{inputenc}
\usepackage{mathtools}
\usepackage{amssymb}
\usepackage{graphicx}
\usepackage{caption}

\usepackage{amsthm}

\newtheorem{preuve}{Preuve}


\font\myfont=cmr12 at 30pt
\title{{\myfont Modèle de battement du cœur}}
\author{Clément GILLI, Louis-Alexis PENELOUX }
\date{Novembre 2022}

\begin{document}

\maketitle

\section{Résumé}

\[\]

On étudie les phénomènes électriques accompagnant le battement cardiaque, ou
plus précisément, le comportement électrique d’une cellule cardiaque, en essayant de mettre en
évidence des propriétés mathématiques significatives. On donne une définition générale des
phénomènes oscillatoires avec relaxation. On propose un modèle analogique de la contraction
d’une cellule du cœur comme exemple d’un phénomène oscillatoire avec relaxation. Ce
modèle conduit à l’équation de Van der Pol, dont on rappelle des propriétés qualitatives. On
insiste sur le problème de la validation du modèle en soulignant les relations avec les résultats
de Van der Pol et Van der Mark au début du vingtième siècle.

\section{Modèle du battement du cœur et oscillations avec relaxation}

\[\]

On veut modéliser les battements du cœur. Depuis la fin du dix-neuvième siècle, on sait que l’activité cardiaque est associée à la production d’une quantité de courant électrique.
D'après les observations physiques, on sait que le phénomène de battement de coeurs fait partie des systèmes naturels auto-excités, dont on remarque des oscillations avec relaxation. Un exemple de système auto-excité est donné par les circuits éléctriques  RLC. On décrit rapidement ce modèle, puis on montrera qu'un habile choix de $\varepsilon$ peux alors ajouter des oscillations avec relaxation à notre système.


\begin{equation}
   \left\{
   \begin{aligned}
        x'(t) &= \varepsilon (y(t) - f(x(t)))\\   
        y'(t) &= -x(t)\\
   \end{aligned}
   \right.
\end{equation}
avec $\varepsilon \in \mathbb{R}$


Si on choisit \(f(x) = \frac{1}{3} x^3 - x\), le système (1) est appelé système de Van der Pol. En dérivant $x'$ et en remplaçant $y'$ par son expression donnée dans le système, on obtient l'équation :

\begin{equation}
    \frac{1}{\varepsilon} x''(t) + (x(t)^2 -1)x'(t) + x(t) = 0
\end{equation}

On peut montrer que (1) et (2) sont bien équivalent, $i.e.$ toute solution de (1) est solution de (2), et réciproquement.

En effet, (1) s'écrit
\begin{equation*}
    \left\{
    \begin{aligned}
        x'(t) &= \varepsilon (y(t) - \frac{1}{3} x(t)^3 + x(t)) \\   
        y'(t) &= -x(t)\\
   \end{aligned}
   \right.
\end{equation*}
dont on dérive la première ligne et remplace \(y'(t) \) par \( -x(t) \) pour obtenir le résultat voulu.
Réciproquement, supposons que \(x\) respecte (2), et que de plus elle admette une primitive sur $\mathbb{R}$. Posons alors $y$ l'unique primitive de $-x$ valant $\frac{x'(0)}{\varepsilon} 
+ \frac{x(0)^3}{3} - x(0)$ en $0$.
On a $y'(t) = -x(t)$, donc en injectant dans (2),

\begin{equation*}
    x''(t) = \varepsilon (x'(t) - x'(t)x(t)^2 + y'(t)) \\
\end{equation*}
En intégrant ces deux quantités, on obtient
\begin{equation*}
    x'(t) = \varepsilon (x(t) - \frac{x(t)^3}{3} + y(t)) + C \\
\end{equation*}
avec $C$ la constante d'intégration. En évaluant en 0, on obtient bien $C = 0$. La première ligne est respectée et l'équivalence est montrée.

\medskip

On remarque qu'on a ajouté une perturbation au coefficient de $x'$ car sinon on aurait une solution sinusoïdale, ce qui ne va pas avec le modèle souhaité. En effet, si le coefficient de $x'$ est nul, et si on pose $\varepsilon = 1$, (les autre cas étant tout aussi simples), l'équation s'écrit :

\begin{equation*}
    x''(t) + x(t) = 0
\end{equation*}

On peut facilement résoudre cette équation, et on obtient :
\[ x(t) = \lambda \cos t + \mu \sin t\]
avec $\lambda ,\mu \in \mathbb{R}$

À partir de là, on peut primitiver $x$ pour trouver $y$ à partir de (1) :

\[ y(t) = -\lambda \sin t + \mu \cos t\]

On remarque notamment que pour tout $t$,  $x(t)^2 + y(t)^2 = 1$. La solution a une allure de cercle : on peut tracer $y(t)$ en fonction de $x(t)$ (on pose $\lambda = \mu = 1$)  $(Figure$ 1$)$ 

\begin{figure}[!h]
\centering
\includegraphics[scale=0.5]{../images/plot_circ.png}
\caption{Graphe de $y(t)$ en fonction de $x(t)$}
\end{figure}

Ce n'est pas satisfaisant. On est obligé de rajouter cette perturbation pour construire un modèle cohérent avec les données physiques.

\section{Étude qualitative de l’équation de Van der Pol}

\[\]

Dans cette section, on décrit brievement quelques propriétés du système de Van der Pol, avec $\varepsilon = 1$ fixé :

\begin{equation}
    \left\{
    \begin{aligned}
        x'(t) &= \varepsilon (y(t) - \frac{1}{3} x(t)^3 + x(t)) \\   
        y'(t) &= -x(t)\\
   \end{aligned}
   \right.
\end{equation}

\begin{enumerate}
    \item Le seul équilibre de (3) est $(0, 0)$ et c’est une source
    \item Le problème de Cauchy pour (3) a une unique solution pour toute donnée initiale
    \item Les trajectoires tournent dans le sens des aiguilles d’une montre autour de l’origine
    \item Il y a une unique solution périodique non triviale que l’on appelle $\gamma$ 
    \item Les autres trajectoires (non triviales) s’approchent de $\gamma$ en tournant
\end{enumerate}

\medskip

\begin{preuve}

\medskip

Le système est en équilibre si et seulement si $x$ et $y$ sont constantes, $i.e.$ 
si et seulement si $x'$ et $y'$ sont nulles, $i.e.$ 
si et seulement si $ \forall t,$ $x(t)=0$ et $y(t)=0$. Donc le seul équilibre de (3) est $(0,0)$.
\\À présent, on étudie le comportement du système (3) linéarisé pour voir le comportement des points d'équilibre, $i.e. (0,0)$ ici :

\begin{equation*}
    \left\{
    \begin{aligned}
        x'(t) &= y(t) + x(t) \\   
        y'(t) &= -x(t)\\
   \end{aligned}
   \right.
\end{equation*}

On pose $u =\begin{pmatrix}x(t)\\y(t)\end{pmatrix}$ et $A=\begin{pmatrix}1 & 1\\-1 & 0\end{pmatrix}$. On a donc $u' = Au$
\smallskip
\\Or $P_A(X)=\begin{vmatrix}1-X & 1\\-1 & -X\end{vmatrix} = X^2-X+1$
\smallskip
\\Donc $Sp_\mathbb{C}(A) = \{ \frac{1}{2} - i \frac{\sqrt{3}}{2}, \frac{1}{2} + i \frac{\sqrt{3}}{2} \} = \{z_1,z_2\}$
\smallskip
\\On a donc $Re(z_1)>0$ et $Re(z_2)>0$, ainsi que $z_1 = \bar{z_2} \implies (0,0)$ est une source ! $\square$   

\end{preuve}

\begin{preuve}
On peut réécrire le système (3) sous forme de problème de Cauchy $X' = f(t,X) $ avec $X=(x(t),y(t))$
et $f:\mathbb{R}^3 \to \mathbb{R}^2$ définie par

\begin{equation}
    f(t,(x,y)) = (y - \frac{1}{3} x^3 + x,-x)
\end{equation}

Or $f$ est clairement $\mathcal{C}^1$, donc elle est lipzitchienne par rapport à sa seconde variable, ici $(x,y)$.
\\Ainsi, d'après le théorème de Cauchy-Lipschitz, (3) a une unique solution pour toute donnée initiale. $\square$
\end{preuve}

\begin{preuve}

Cette fois ci, on va prouver cette propriété numériquement. 
On a donc résolu numériquement le système (3), puis tracé différentes solutions à partir de conditions initiales $(x_0,y_0)$ différentes $(Figure$ 2$)$.
On a également tracé le champs de vecteur associé. On constate que peu importe le point $(x_0,y_0)$ de départ, la courbe finit toujours par tourner autour de l'origine dans le sens des aiguilles d'une montre.
Le champs de vecteur permet de bien visualiser le sens de rotation. $\square$

\begin{figure}[!h]
\centering
\includegraphics[scale=0.4]{../images/plot_solh.png}
\caption{Graphe de $y(t)$ en fonction de $x(t)$}
\end{figure}

\end{preuve}

On admettra la preuve des autres propriétés. On peut cependant utiliser le schéma de Runge Kutta pour montrer les résultats attendus $(Figure$ 3$)$.

\begin{figure}[!h]
\centering
\includegraphics[scale=0.4]{../images/dphase.png}
\caption{Diagramme des phases et intensité du courant en fonction du temps}
\end{figure}

\newpage
\section{Choix du paramètre $\varepsilon$}

\[\]

On essaye de choisir $\varepsilon$ pour obtenir un comportement oscillatoire avec relaxation, avec pour
but principal d’obtenir des oscillations sensiblement non sinusoïdales. Pour cela, on remarque
que si $0 <\varepsilon < 1$, on obtient numériquement des courbes presque sinusoïdales. Si
$\varepsilon \leq 0$, on peut
montrer que le comportement qualitatif des solutions n’est pas celui recherché.
On est donc amené à considérer des valeurs positives et élevées du paramètre $\varepsilon$. 
Par exemple, pour $\varepsilon = 100$, on a le comportement décrit par la $Figure$ 4.
Dans le diagramme des phases de la $Figure$ 4, on peut aussi remarquer que quand $\varepsilon$ est grand,
certaines parties de la trajectoires et de la courbe $y = f (x)$ sont très proches l’une de l’autre.
Cela est cohérent avec l’équation (1). On va donc choisir $\varepsilon$ positif et ”grand”, (dans les exemples, on gardera
$\varepsilon$ = 100).

\begin{figure}[!h]
    \centering
    \includegraphics[scale=0.4]{../images/epsilon_choice.png}
    \captionsetup{justification=centering,margin=2cm}
    \caption{Diagramme des phases et intensité du courant en fonction du temps avec $\varepsilon = 100$}
\end{figure}

\section{Période et amplitude asymptotique}

\[\]

On veut maintenant estimer la période fondamentale $T$ de la solution. Pour cela, on remarque qu'analytiquement c'est assez complexe, c'est pourquoi on décide de l'estimer numériquement.
L'algorithme est assez simple en pratique: On cherche le point où la valeur de l'ordonnée est maximale (une fois que la solution s'est stabilisée) puis on cherche le prochain point où l'ordonnée est maximale.
En faisant la différence des abscisses, on obtient une approximation de la période.
En faisant tendre $\varepsilon$ vers $+\infty $, le résultat analtyique (admis ici) nous dit que $\lim_{\varepsilon \to \infty} T = 3 - 2\ln 2 \simeq 1.61$.
Avec notre algorithme assez naïf, on arrive à trouver $T \simeq 1.66$ pour $\varepsilon$ assez grand.

\begin{figure}[!h]
    \centering
    \includegraphics[scale=0.5]{../images/periode.png}
    \captionsetup{justification=centering,margin=2cm}
    \caption{Intensité du courant en fonction du temps avec $\varepsilon = 100$. La période estimée est $T \simeq 1.9$}
\end{figure}

\section{Extensions}

\[\]

\end{document}